\documentclass[1pt]{article}
\usepackage{tabularx}
\usepackage{graphicx}
\usepackage{mathptmx}
\usepackage[11pt]{moresize}
\graphicspath{ {C:/Users/Nischith/Pictures/} }
\begin{document}
	\begin{center}
	{\large \textbf{Nischith B.O}}\\
		\noindent\hrulefill\\
	\end{center}
	Veerabhadhreshwara Nilaya, \hfill Contact: 8495862194\\
	Vivekanandha Badavane, \hfill email-id: nischith23gowda@gmail.com\\
    behind Chowdeshwari Devastana,\hfill \\ Shivamoga-577205,\hfill \\ Karnataka.\\
	
	\hfill \includegraphics[width=3cm, height=3cm]{es}\\
	
	\begin{flushleft}
			
			\hspace{0.2cm}\textbf{OBJECTIVE : }\\
		    	To work in the field related to real time image processing,machine learning  and micro controller.
		    
				
			\hspace{1cm}\\
			\hspace{1cm}\\
			
			\hspace{0.2cm}\textbf{EDUCATION : }\\
			
			%\resizebox{13cm}{!}{
			\begin{center}
				\begin{tabular}{ |c|c|c|c|c| } 
					\hline
					Degree & School/College & University & Passing Year & Pass Percentage\\ 
					\hline
					Matriculation(10th) & Jnanadeepa school & CBSE Board & 2014 & 10 GPA\\ 
					\hline
					Intermediate(12th) & Aurobindo PU College & State Board & 2016 & 94\% \\ 
					\hline
					B.TECH & PES & PES University & 2020 & present - 8.92 CGPA\\
					\hline
				\end{tabular}
			\end{center}
			%}
		
		\hspace{1cm}\\
		\hspace{1cm}\\
		
		\hspace{0.2cm}\textbf{Projects : } \\
		\ \\
		\hspace{.8cm} 1. Public Garden Automation System using 8051uc.\\
		\hspace{5.6cm}\\Project to build a system that monitors the external environment
		continuously and takes necessary action to normalize the condition.
		Monitoring of the moisture content of the soil at regular intervals
		and accordingly open or close the solenoid valve to water the
		garden, inputs from the moisture sensor was fed to 8051 micro
		controller which was programmed using embedded C to turn on the
		valve if necessary using relay.
		Monitoring the garden lights using LDR sensor.
		Automatic gate opening system using ultrasonic sensor display of
		time on LCD. \\
		\ \\
		\hspace{.8cm} 2. c program to implement karastuba algorithm .\\ \ \\
		implementation of long multiplication on numbers read as strings\\
		\ \\
		\hspace{.8cm} 3. PBL - Configuring routing protocols(RIP and OSPF) using GNS3.
		
		\hspace{1cm}\\ \hspace{1cm}
		
		{\small \textbf{TRAINING \& INTERNSHIPS:}}\\ \ \\
		\textbullet \ \ Minors in computer science.\\
		\textbullet \ \ Delegate of Young India Challenge \\	
		\textbullet \ \ Intern at sprouts solution.\\
		
		\hspace{1cm}\\ \hspace{1cm}
		
		{\small \textbf{RESEARCH AND PUBLICATIONS:}}\\ \ \\
		1. \ \ None.\\
		2. \ \ None.\\
		
		\hspace{1cm}\\ \hspace{1cm}
		
		{\small \textbf{TECHNICAL SKILLS :}}\\ \ \\
		\textbullet \ \  Programming Skills :  \\ \hspace{1.4cm}C, Python,django basics ,HTML/CSS basics Scilab, Matlab basics, Vivado, QUCS and Proteus simulations,Sublime.\\
		
		\textbullet \ \ Machine Learning :  \\ \hspace{1.4cm}Pytorch, Tensorflow basics.\\
		
		
		\hspace{1cm}\\ \hspace{1cm}
		
		
		{\small \textbf{SOFT SKILLS :}}\\ \ \\
		1.\ \ Problem-solving skills.\\
		2.\ \  Leadership.\\	
		3.\ \ Teamwork.\\
		4.\ \ Responsibility.\\
		5.\ \ Conflict resolution\\
		6.\ \ Ability to work under pressure\\
		
			\hspace{1cm}\\ \hspace{1cm}
		
		
		{\small \textbf{EXTRA-CURRICULAR ACTIVITIES :}}\\ \ \\
		\textbullet \ \ sports:\\ \hspace{1cm}captain of school and college cricket team, football.\\
		\textbullet \ \ Community service.\\
		
		
		\hspace{1cm}\\ \hspace{1cm}
		
		
		{\small \textbf{CO-CURRICULAR ACTIVITIES :}}\\ \ \\
		1.\ \ e-yantra Robotics competition (2017-18),(2018-19).\\	
		2.\ \ JED-I project to build a Fruit plucking Robot using Arduino.\\
		
		
		\hspace{1cm}\\ \hspace{1cm}
		
		
		{\small \textbf{ Personal Details :}}\\
		\hspace{1cm}Father's Name \ :Omprakash B.P.K\\	
		\hspace{1cm}Mother's Name : Bharathi S.S\\
		\hspace{1cm}Sex : Male\\	
		\hspace{1cm}Date of Birth : 23-04-1998\\
		\hspace{1cm}Nationality : Indian\\
		\hspace{1cm}Marital Status : Not married\\
		
		\hspace{1cm}\\ \hspace{1cm}
		
		{\small \textbf{Reference :}}\\
			
	\end{flushleft}
	
\end{document}